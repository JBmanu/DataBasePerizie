\documentclass[a4paper,12pt]{report}

\usepackage{tabularx}
\usepackage{alltt, fancyvrb, url}
\usepackage{graphicx}
\usepackage[utf8]{inputenc}
\usepackage{float}
\usepackage{hyperref}
\usepackage{caption}


% Questo commentalo se vuoi scrivere in inglese.
\usepackage[italian]{babel}

\usepackage[italian]{cleveref}

% Centra verticalmente i contenuti delle righe
\renewcommand\tabularxcolumn[1]{m{#1}}


\title{\textbf{Elaborato per il corso Basi di dati}
Progettazione di una base di dati per la gestione di video perizie}


\author{
\\Buizo Manuel
\\Matteini Mattia
\\Paganelli Alberto
}
\date{\today}


\begin{document}

\maketitle

\tableofcontents

\chapter{Analisi dei requisiti}

Si vuole realizzare un database a supporto della gestione di video perizie (perizie a distanza tramite videochiamata) effettuate per varie assicurazioni italiane.
\\
La base di dati dovrà immagazzinare informazioni relativo a tutto l’ambito assicurativo, dalle assicurazioni, agli studi peritali che effettueranno le perizie.
\\
Si dovrà tenere conto anche di polizze, assicurati, dipendenti, documenti relativi alle video-perizie.


\section{Intervista }

Si vuole tenere traccia di tutte le video-perizie effettuate, di ogni studio peritale e delle parti coinvolte. 
\\
Ogni assicurazione dovrà generare sinistri di varia natura (R.C.A., Furto, Incendio, ecc…).
\\
Questi verranno assegnati agli studi, i quali si occuperanno di portare a termine le attività peritali.
\\
Ogni studio peritale dispone di un supervisore che avrà il compito di ricevere i sinistri che arrivano dalle assicurazioni e di smistarli ai periti del proprio studio.
\\
Il supervisore dello studio quindi creerà un incarico relativo al sinistro arrivatogli e lo assegnerà ad un perito che dovrà svolgere le attività peritali inerenti (video-perizia, richiesta documenti, ecc.. ).
\\
Ogni incarico conterrà informazioni riguardanti sinistro di riferimento, perito incaricato e stato di avanzamento (Aperto, Chiuso).
\\
Inoltre includerà uno storico delle video-perizie svolte e  l’insieme dei documenti richiesti all’assicurato (come eventuali contratti o documenti personali).
\\
Il perito quando riceverà l'incarico dal supervisore, dovrà mettersi in contatto con l'assicurato al fine di definire i dettagli per l'effettuazione della video-perizia ed eventualmente richiedere dei documenti per le pratiche preliminari.
\\
Durante una video-perizia si potranno raccogliere vari media (foto e video) che saranno allegati alla perizia.
\\
Ogni media a sua volta è comprensivo di metadati ricavati dal GPS del dispositivo dell’assicurato.
\\
Per ogni documento invece dovremo sapere la sua tipologia e se sarà necessaria o meno la firma dell’assicurato.
\\
In questo modo teniamo traccia di ogni sinistro, dall’assicurazione ai tipi di documenti richiesti o alle foto georeferenziate scattate durante la video-perizia.


\section{Definizione delle specifiche in linguaggio naturale ed estrazione dei concetti principali}
\subsection{Glossario dei termini}
\mbox{}\\
\def\arraystretch{2}% 
\begin{tabularx}{\textwidth}{ m{3cm} | m{6cm} | m{3cm}}
    \textbf{Termine} & \textbf{Descrizione} & \textbf{Sinonimo} \\
\hline
Assicurazione & Colei che riceve da cittadini nuove richieste e crea sinistri & Ente esterno\\ \hline
Studio & Colei che riceve da cittadini nuove richieste e crea sinistri & Ente esterno\\ \hline
Supervisore & Colui che, all’interno dello studio, ha il compito di generare incarichi e assegnarli ad uno dei propri periti & Coordinatore, Direttore\\ \hline
Perito & Colui che si occupa della attività peritali & Incaricato dal supervisore, membro dello studio/ufficio\\ \hline
Assicurato & Colui che farà parte alla video-perizia e che si occupa della ripresa del sinistro & Cliente, parte coinvolta\\ \hline
Sinistro & Danno e relativo tipo di danno da periziare & \\ \hline
Incarico & Insieme di attività peritali atte all’intero svolgimento della perizia & Fascicolo\\ \hline
Video-perizia & Perizia eseguita telematicamente tramite smartphone o dispositivo mobile & Perizia telematica, videochiamata
\\ 

\end{tabularx}
\noindent
\def\arraystretch{2}% 
\begin{tabularx}{\textwidth}{ m{3cm} | m{6cm} | m{3cm}}
Documenti & Documenti richiesti al fine di eseguire una perizia completa & \\ \hline
Media & Media raccolti durante la video-perizia, comprensivi di metadati & Foto, video\\ \hline
Metadati & Informazioni raccolte dal dispositivo dell’assicurato, in questo caso specifico quelli relativi alla geolocalizzazione & \\
\end{tabularx}
\\
\\

\subsection{Riassunto dei concetti principali}

Ogni Sinistro, è individuato tramite un identificativo incrementale e può essere di un solo \texttt{TIPO\_SINISTRO}.
\\
Un’ Assicurazione, identificata tramite la sua denominazione, cede la gestione del \texttt{SINISTRO} allo \texttt{STUDIO} (identificato dalla P.Iva e da un identificativo incrementale) e lo stesso sinistro non può essere assegnato ad un altro ufficio.
\\
Ogni studio può avere più di un \texttt{SUPERVISORE} ma ne ha almeno uno.
\\
Gli \texttt{STUDI} possono ricevere sinistri da tutte le \texttt{ASSICURAZIONI} e un \texttt{SUPERVISORE} deve  creare un \texttt{INCARICO} e assegnarne la gestione ad un proprio \texttt{PERITO}.
\\
Non può essere assegnato un \texttt{INCARICO} a più di un \texttt{PERITO} ma a un \texttt{PERITO} possono essere assegnati più \texttt{INCARICHI} e da più \texttt{SUPERVISORI}.
\\
Il \texttt{PERITO}, che avrà il compito di svolgere le attività peritali inerenti all’\texttt{INCARICO} a lui assegnato, dovrà poter svolgere anche più di una \texttt{VIDEO-PERIZIA} per entrare in contatto con l’\texttt{ASSICURATO} e poter scrivere una descrizione del danno (ad esempio se si vede meglio in altra fase della giornata, danno grande che richiede più videochiamate, ecc...  ).
\\
Potrà anche lavorare a più \texttt{INCARICHI} alla volta e nello stesso giorno.
\\
L’\texttt{ASSICURATO} invece sarà registrato tramite un’anagrafica ed identificato mediante il codice fiscale.
\\
La \texttt{VIDEO-PERIZIA} e i \texttt{MEDIA} raccolti, sono relativi esclusivamente ad un \texttt{INCARICO}. 
\\
Ogni \texttt{INCARICO} possiede anche una raccolta di \texttt{DOCUMENTI} riguardanti il sinistro.
\\
I \texttt{DOCUMENTI} riguardano principalmente l’assicurato e il tipo di sinistro, non sapendo quindi quanti documenti possono essere richiesti, non vi è nessun vincolo. 
\\
Ogni \texttt{VIDEO-PERIZIA} deve comprendere anche un luogo effettivo e sarà quindi localizzato tramite coordinate e sistema di riferimento. (Altitudine, Latitudine, Longitudine e Est o Ovest).
\\
Ogni \texttt{VIDEO-PERIZIA} ed ogni \texttt{MEDIA} deve essere localizzato per essere definita valida.
\\
\\
\textbf{Segue un elenco delle principali azioni richieste:}
\begin{enumerate}
    \item \textsc{Inserire un assicurato}
    \item \textsc{Visualizzare tutte le polizze di un assicurato}
    \item \textsc{Stipulazione di una nuova polizza tra assicurato e assicurazione}
    \item \textsc{Registrare un nuovo studio peritale}
    \item \textsc{Rimuovere uno studio peritale}
    \item \textsc{Generare un sinistro e delegarlo a uno studio peritale}
    \item \textsc{Assumere un nuovo perito o supervisore in uno studio}
    \item \textsc{Licenziare il perito di uno studio}
    \item \textsc{Il supervisore crea un nuovo incarico e lo assegna a un perito}
    \item \textsc{Leggere tutti gli incarichi aperti in un determinato studio}
    \item \textsc{Visualizzare quale assicurato ha svolto una determinata video-perizia}
    \item \textsc{Aggiungere un documento ad un incarico}
    \item \textsc{Inserire una video-perizia per un incarico}
    \item \textsc{Visualizzare il proprietario (assicurato) di un documento}
    \item \textsc{Visualizzare in media quanto durano le video-perizie di un determinato studio peritale}
    \item \textsc{Qual è la provincia nella quale avvengono più sinistri}
\end{enumerate}


\chapter{Progettazione Concettuale}

\section{Schema scheletro}

\textsc{Seguiranno ora dei sotto-schemi E/R iniziali, che verranno raffinati nella sezione successiva}

\clearpage

Dal dominio identificato in seguito all’intervista si può notare che sono presenti tre entità rappresentanti delle persone: \textbf{Perito}, \textbf{Supervisore} e \textbf{Assicurato}.
\\
Queste entità quindi sono generalizzate da \textbf{Persona}, che raccoglie gli attributi in comune.
\\
In seguito si ha una collaborazione tra \textbf{Studi Peritali} e \textbf{Assicurazioni}, uno studio può non avere ancora collaborazioni con le assicurazioni (perché magari creato da poco) e può averne molteplici, invece un’assicurazione deve collaborare con almeno uno studio perché altrimenti non potrebbe lavorare e non avrebbe senso di esistere in questa modellazione.
\\
Le assicurazioni sono identificate dalle proprie denominazioni, mentre gli studi da un ID. 
%poiché è difficile distinguerle per sedi o per partite IVA (una singola P. IVA ) potrebbe avere più studi.
\\
Uno studio peritale può assumere sia supervisori (almeno uno altrimenti nessuno smisterebbe gli incarichi) sia periti, abbiamo quindi un associazione ternaria che andremo a rifinire in seguito.
\\
\begin{figure}[ht]
    \begin{center}
        \centering
        \includegraphics[width=\textwidth]{img/StudioPeritale.png}
    \end{center}
\end{figure}
\clearpage

Un \textbf{Perito}, in un determinato momento, può svolgere zero incarichi, ma anche molteplici, invece un \textbf{Incarico} può essere svolto solo da un perito.
\\
L’incarico deve per forza avere almeno una \textbf{Video Perizia} annessa (per la corretta esecuzione delle pratiche peritali), inoltre è identificato dalla combinazione di codice e perito.
\\
La video perizia può essere eseguita solo in presenza di un incarico, e quindi è identificata da un numero incrementale combinato al codice dell’incarico corrispondente.
\\
Durante la videochiamata possono essere scattate \textbf{Foto} e registrati \textbf{Video}, ma non è sempre necessario.
\\
Questi ultimi sono entità figlie di \textbf{Media}, hanno una dimensione e sono identificati dal nome e dall’estensione del file.
\\
\begin{figure}[ht]
    \begin{center}
        \centering
        \includegraphics[width=\textwidth]{img/VideoPerizia.png}
    \end{center}
\end{figure}
\clearpage

\section{Raffinamenti proposti}

\section{Schema concettuale finale}

\begin{figure}[H]
\centering{}
\includegraphics[width=.7\textwidth]{img/observer}
\caption{Il pattern Observer è usato per consentire a GLaDOS di informare tutti i sistemi di output in ascolto}
\label{img:observer}
\end{figure}



\begin{figure}[h]
\centering{}
\includegraphics[width=\textwidth]{img/badarch}
\caption{Schema UML mal fatto e con una pessima descrizione, che non aiuta a capire. Don't try this at home.}
\label{img:badarch}
\end{figure}


\chapter{Progettazione Logica}
\section{Stima del volume dei dati}

\mbox{}\\
\def\arraystretch{2}% 
\begin{tabularx}{\textwidth}{ p{6cm} | >{\centering\arraybackslash}p{2cm} | >{\centering\arraybackslash}X }
    \textbf{Concetto} & \textbf{Costrutto} & \textbf{Volume} \\
\hline
ASSICURAZIONE & E & 30\\ \hline
EROGAZIONE & A & 2.000.000\\ \hline
POLIZZA & E & 2.000.000\\ \hline
SPECIFICAZIONE & A & 2.000.000\\ \hline
TIPO\_POLIZZA & E & 15\\ \hline
STIPULAZIONE & A & 2.000.000\\ \hline
ASSICURATO & E & 1.000.000\\ \hline
LUOGO & E & 100.000\\ \hline
AVVENIMENTO & A & 100.000\\ \hline
RESIDENZA & A & 3.000\\ \hline
COINVOLGIMENTO & A & 100.000\\
\end{tabularx}

\noindent
\def\arraystretch{2}% 
\begin{tabularx}{\textwidth}{ p{6cm} | >{\centering\arraybackslash}p{2cm} | >{\centering\arraybackslash}X }
GENERAZIONE & A & 100.000\\ \hline
SINISTRO & E & 100.000\\ \hline
CATEGORIZZAZIONE & A & 100.000\\ \hline
CATEGORIA\_SINISTRO & E & 20\\ \hline
DELEGAZIONE & A & 100.000 \\ \hline
STUDIO\_PERITALE & E & 3.000\\ \hline
SUPERVISORE & E & 4.000\\ \hline
SUPERVISIONE & A & 4.000\\ \hline
PERITO & E & 60.000\\ \hline
ASSUNZIONE & A & 60.000\\ \hline
ASSEGNAZIONE & E & 100.000\\ \hline
RICEZIONE & A & 100.000\\ \hline
INCARICO & E & 100.000\\ \hline
FASCICOLO & A & 100.000\\ \hline
DOCUMENTO & E & 300.000\\ \hline
ANNESSIONE & A & 120.000\\ \hline
VIDEO\_PERIZIA & E & 120.000\\ \hline
ALLEGATO & A & 50.000\\ \hline
MEDIA & E & 70.000\\
\end{tabularx}

\clearpage
\section{Descrizione delle operazioni principali e stima della loro frequenza}

\def\arraystretch{2}% 
\begin{tabularx}{\textwidth}{ >{\centering\arraybackslash}p{2cm} | X |  >{\centering\arraybackslash}p{3cm} }
    \textbf{Codice} & \textbf{Operazione} & \textbf{Frequenza}\\
\hline
1 & Inserire un assicurato & 300 al giorno\\ \hline
2 & Visualizzare tutte le polizze di un assicurato & 20 al giorno\\ \hline
3 & Stipulazione di una nuova polizza tra assicurato e assicurazione & 1.000 al giorno\\ \hline
4 & Registrare un nuovo studio peritale & 15 al mese\\ \hline
5 & Rimuovere uno studio peritale & 15 al mese\\ \hline
6 & Generare un sinistro e delegarlo a uno studio peritale & 25.000 al giorno\\ \hline
7 & Assumere un nuovo perito o supervisore in uno studio & 5.000 all'anno\\ \hline
8 & Licenziare il perito di uno studio & 1.500 all'anno\\ \hline
9 & Il supervisore crea un nuovo incarico e lo assegna a un perito & 25.000 al giorno\\ \hline
10 & Leggere tutti gli incarichi aperti in un determinato studio & 2.000 al giorno\\ \hline
11 & Visualizzare quale assicurato ha svolto una determinata video-perizia & 15.000 al giorno\\ 

\end{tabularx}

\noindent
\def\arraystretch{2}% 
\begin{tabularx}{\textwidth}{ >{\centering\arraybackslash}p{2cm} | X |  >{\centering\arraybackslash}p{3cm} }
12 & Aggiungere un documento ad un incarico & 30.000 al giorno\\ \hline
13 & Inserire una video-perizia per un incarico & 30.000 al giorno\\ \hline
14 & Visualizzare il proprietario (assicurato) di un documento & 25.000 al giorno\\ \hline
15 & Visualizzare in media quanto durano le video-perizie di un determinato studio peritale & 20 al giorno\\ \hline
16 & Visualizzare a quale sinistro è associato un determinato incarico & 10.000 al giorno
\end{tabularx}
\\
\\
\section{Schemi di navigazione e tabelle degli accessi}
In questa sezione verranno descritte le operazioni che non usufruiscono di ridondanze, per quelle con esse vedere il capitolo successivo.


\subsection{1 - Inserire un assicurato}
Inserire un assicurato implica che abbia stipulato un contratto con un assicurazione.
\\
\\
\def\arraystretch{2}% 
\begin{tabularx}{\textwidth}{ >{\centering\arraybackslash}p{3cm} | >{\centering\arraybackslash}X | >{\centering\arraybackslash}X |  >{\centering\arraybackslash}X }
    \textbf{Concetto} & \textbf{Costrutto} & \textbf{Accessi} & \textbf{Tipo} \\
\hline
Assicurato & E & 1 & S \\
Stipulazione & A & 1 & S \\
Polizza & E & 1 & S \\
Erogazione & A & 1 & S \\
\end{tabularx}
\begin{center}
\textbf{Totale : 4S * 300 al giorno = 2.400 al giorno}
\end{center}

\clearpage
\subsection{2 - Visualizzare tutte le polizze di un assicurato}
In media ogni assicurato avrà due polizze.
\\
\\
\def\arraystretch{2}% 
\begin{tabularx}{\textwidth}{ >{\centering\arraybackslash}p{3cm} | >{\centering\arraybackslash}X | >{\centering\arraybackslash}X |  >{\centering\arraybackslash}X }
    \textbf{Concetto} & \textbf{Costrutto} & \textbf{Accessi} & \textbf{Tipo} \\
\hline
Assicurato & E & 1 & L \\
Stipulazione & A & 1 & L \\
Polizza & E & 2 & L \\
\end{tabularx}
\begin{center}
\textbf{Totale : 4L * 20 al giorno = 80 al giorno}
\end{center}

\subsection{3 - Stipulazione di una nuova polizza tra assicurato e assicurazione}

\def\arraystretch{2}% 
\begin{tabularx}{\textwidth}{ >{\centering\arraybackslash}p{3cm} | >{\centering\arraybackslash}X | >{\centering\arraybackslash}X |  >{\centering\arraybackslash}X }
    \textbf{Concetto} & \textbf{Costrutto} & \textbf{Accessi} & \textbf{Tipo} \\
    \hline
    Stipulazione & A & 1 & S \\
    Polizza & E & 1 & S \\
    Erogazione & A & 1 & S \\
\end{tabularx}
\begin{center}
\textbf{Totale : 3S * 1.000 al giorno = 6.000 al giorno}
\end{center}
\clearpage
\subsection{4 - Registrare un nuovo studio peritale}
Inserire uno studio peritale ci vincola ad aggiungere anche almeno un dipendente e un supervisore, altrimenti lo studio non potrebbe assegnare lavori e svolgere incarichi, ciò vuol dire che si dovrà accedere in scrittura anche in Perito, supervisore, e relative associazioni.
\\
\\
\def\arraystretch{2}% 
\begin{tabularx}{\textwidth}{ >{\centering\arraybackslash}p{3cm} | >{\centering\arraybackslash}X | >{\centering\arraybackslash}X |  >{\centering\arraybackslash}X }
    \textbf{Concetto} & \textbf{Costrutto} & \textbf{Accessi} & \textbf{Tipo} \\
    \hline
    Studio Peritale & E & 1 & S \\
    Supervisore & E & 1 & S \\
    Supervisione & A & 1 & S \\
    Assunzione & A & 1 & S \\
    Perito & E & 1 & S \\
\end{tabularx}
\begin{center}
\textbf{Totale : 5S * 15 al mese = 150 al mese}
\end{center}

\clearpage
\subsection{5 - Rimuovere uno studio peritale}
Rimuovere uno studio peritale implica che non si deve tenere più traccia dei dipendenti, dei lavori e degli incarichi dello studio, inoltre, deve essere eliminata la delegazione del sinistro, in modo tale che possa essere ri-delegato a un altro studio.
\\
Uno studio peritale in media ha 33 sinistri e incarichi, 20 periti e 1 supervisore.
\\
\\
\def\arraystretch{2}% 
\begin{tabularx}{\textwidth}{ >{\centering\arraybackslash}p{3cm} | >{\centering\arraybackslash}X | >{\centering\arraybackslash}X |  >{\centering\arraybackslash}X }
    \textbf{Concetto} & \textbf{Costrutto} & \textbf{Accessi} & \textbf{Tipo} \\
    \hline
    Studio Peritale & E & 1 & S \\
    Supervisore & E & 1 & S \\
    Supervisione & A & 1 & S \\
    Assunzione & A & 20 & S \\
    Perito & E & 20 & S \\
    Delegazione & A & 33 & S \\
    Incarico & E & 33 & S \\
    Svolgimento & A & 33 & S \\
    Assegnazione & A & 33 & S \\
    Ricezione & A & 33 & S \\
\end{tabularx}
\begin{center}
\textbf{Totale : 241S * 15 al mese = 7.230 al mese}
\end{center}
\clearpage

\subsection{6 - Generare un sinistro e delegarlo a uno studio peritale}
Quando di genera un sinistro si deve per forza specificare anche la categoria.
\\
\\
\def\arraystretch{2}% 
\begin{tabularx}{\textwidth}{ >{\centering\arraybackslash}p{3cm} | >{\centering\arraybackslash}X | >{\centering\arraybackslash}X |  >{\centering\arraybackslash}X }
    \textbf{Concetto} & \textbf{Costrutto} & \textbf{Accessi} & \textbf{Tipo} \\
    \hline
    Generazione & A & 1 & S \\
    Sinistro & E & 1 & S \\
    Specificazione & A & 1 & S \\
    Categoria\_Sinistro & E & 1 & S \\
    Delegazione & A & 1 & S \\
\end{tabularx}
\begin{center}
\textbf{Totale : 5S * 25.000 al giorno = 250.000 al giorno}
\end{center}

\subsection{7 - Assumere un nuovo perito o supervisore in uno studio}
Assumere un perito o un supervisore comporta lo stesso numero di accessi per entrambe le operazioni, non c'è nessun vincolo da tenere in considerazione (a parte associarlo allo studio) quindi l'inserimento è immediato.
\\
\\
\def\arraystretch{2}% 
\begin{tabularx}{\textwidth}{ >{\centering\arraybackslash}p{3cm} | >{\centering\arraybackslash}X | >{\centering\arraybackslash}X |  >{\centering\arraybackslash}X }
    \textbf{Concetto} & \textbf{Costrutto} & \textbf{Accessi} & \textbf{Tipo} \\
    \hline
    Perito / Supervisore & E & 1 & S \\
    Assunzione / Supervisione & A & 1 & S \\
\end{tabularx}
\begin{center}
\textbf{Totale : 2S * 5.000 all'anno = 20.000 all'anno}
\end{center}

\clearpage
\subsection{8 - Licenziare il perito di uno studio}
Quando si licenzia un perito lo si deve aggiornare anche nei lavori a lui assegnati e negli incarichi da lui svolti. 
\\
\\
\def\arraystretch{2}% 
\begin{tabularx}{\textwidth}{ >{\centering\arraybackslash}p{3cm} | >{\centering\arraybackslash}X | >{\centering\arraybackslash}X |  >{\centering\arraybackslash}X }
    \textbf{Concetto} & \textbf{Costrutto} & \textbf{Accessi} & \textbf{Tipo} \\
    \hline
    Perito & E & 1 & S \\
    Ricezione & A & 1 & S \\
    Incarico & E & 1 & S \\
\end{tabularx}
\begin{center}
\textbf{Totale : 4S * 1.500 all'anno = 12.000 all'anno}
\end{center}

\subsection{9 - Il supervisore crea un nuovo lavoro e lo assegna a un perito}

\def\arraystretch{2}% 
\begin{tabularx}{\textwidth}{ >{\centering\arraybackslash}p{3cm} | >{\centering\arraybackslash}X | >{\centering\arraybackslash}X |  >{\centering\arraybackslash}X }
    \textbf{Concetto} & \textbf{Costrutto} & \textbf{Accessi} & \textbf{Tipo} \\
    \hline
    Perito & E & 1 & S \\
    Ricezione & A & 1 & S \\
    Svolgimento & A & 1 & S \\
    Incarico & E & 1 & S \\
\end{tabularx}
\begin{center}
\textbf{Totale : 4S * 25.000 al giorno = 200.000 all'anno}
\end{center}

\clearpage

\subsection{10 - Leggere tutti gli incarichi aperti in un determinato studio}
Considerando che in media ogni studio peritale si occupa di 33 incarichi, e gli stati possono essere due (Aperto e Chiuso), avremo sempre in media 16 letture.
\\
\\
\def\arraystretch{2}% 
\begin{tabularx}{\textwidth}{ >{\centering\arraybackslash}p{3cm} | >{\centering\arraybackslash}X | >{\centering\arraybackslash}X |  >{\centering\arraybackslash}X }
    \textbf{Concetto} & \textbf{Costrutto} & \textbf{Accessi} & \textbf{Tipo} \\
    \hline
    Studio Peritale & E & 1 & L \\
    Incarico & E & 16 & L \\
\end{tabularx}
\begin{center}
\textbf{Totale : 17L * 2.000 al giorno = 34.000 al giorno}
\end{center}


\subsection{12 - Aggiungere un documento ad un incarico}
Un incarico in media include un fascicolo di 3 documenti.
\\
\\
\def\arraystretch{2}% 
\begin{tabularx}{\textwidth}{ >{\centering\arraybackslash}p{3cm} | >{\centering\arraybackslash}X | >{\centering\arraybackslash}X |  >{\centering\arraybackslash}X }
    \textbf{Concetto} & \textbf{Costrutto} & \textbf{Accessi} & \textbf{Tipo} \\
    \hline
    Incarico & E & 1 & L \\
    Documento & E & 3 & S \\
    Fascicolo & A & 3 & S \\
\end{tabularx}
\begin{center}
\textbf{Totale : (1L + 6S) * 35.000 al giorno = 26.000 al giorno}
\end{center}

\subsection{13 - Inserire una video-perizia per un incarico}

\def\arraystretch{2}% 
\begin{tabularx}{\textwidth}{ >{\centering\arraybackslash}p{3cm} | >{\centering\arraybackslash}X | >{\centering\arraybackslash}X |  >{\centering\arraybackslash}X }
    \textbf{Concetto} & \textbf{Costrutto} & \textbf{Accessi} & \textbf{Tipo} \\
    \hline
    Video-Perizia & E & 1 & S \\
    Annessione & A & 1 & S \\
\end{tabularx}
\begin{center}
\textbf{Totale : 2S * 30.000 al giorno = 120.000 al giorno}
\end{center}

\subsection{15 - Visualizzare in media quanto durano le video-perizie di un determinato studio peritale}
In media ogni studio peritale esegue 40 video-perizie, di conseguenza andranno lette tutte per poter accedere alla durata di ognuna e farne una media.
\\
\\
\def\arraystretch{2}% 
\begin{tabularx}{\textwidth}{ >{\centering\arraybackslash}p{3cm} | >{\centering\arraybackslash}X | >{\centering\arraybackslash}X |  >{\centering\arraybackslash}X }
    \textbf{Concetto} & \textbf{Costrutto} & \textbf{Accessi} & \textbf{Tipo} \\
    \hline
    Studio Peritale & E & 1 & L \\
    Incarico & E & 33 & L \\
    Video-Perizia & E & 40 & L \\
    Annessione & A & 40 & L \\
\end{tabularx}
\begin{center}
\textbf{Totale : 114L * 20 al giorno = 2.280 al giorno}
\end{center}

\subsection{16 - Visualizzare a quale sinistro è associato un determinato incarico}

\def\arraystretch{2}% 
\begin{tabularx}{\textwidth}{ >{\centering\arraybackslash}p{3cm} | >{\centering\arraybackslash}X | >{\centering\arraybackslash}X |  >{\centering\arraybackslash}X }
    \textbf{Concetto} & \textbf{Costrutto} & \textbf{Accessi} & \textbf{Tipo} \\
    \hline
    Incarico & E & 1 & L \\
    Sinistro & E & 1 & L \\
\end{tabularx}
\begin{center}
\textbf{Totale : 2L * 10.000 al giorno = 20.000 al giorno}
\end{center}

\clearpage
\section{Raffinamento dello schema}

\subsection{Eliminazione delle gerarchie}
Per l’eliminazione della gerarchia Persona si è scelto di adottare l’approccio del collasso verso il basso, replicando così gli attributi in Assicurato, supervisore e Perito.
Si è adottata questa strategia in quanto si ha la necessità di trattare nello specifico ogni tipo di persona dando rilievo al ruolo di ognuna, per esempio bisogna ben diversificare i ruoli di supervisore e Perito.
\\
\subsection{Eliminazione degli attributi composti}
Nello schema E/R non è presente alcun attributo composto, l'unico che esisteva in fase di progettazione concettuale era Luogo, ma poi si è deciso di elevarlo direttamente ad entità perché era di rilevante importanza sia per Sinistro che per Studio Peritale.
Sarà poi necessario accertarsi, a livello applicativo che gli attributi dell'entità Luogo coesistano senza dare origine a incoerenze (Esempio: non potranno esistere luoghi con un cap non corrispondente a quello della città).
\\
\subsection{Scelta delle chiavi primarie}
Nello schema sono già state specificate tutte le chiavi primarie per la maggior parte delle entità, per quanto riguarda le entità della gerarchia Persona invece, si sceglie di usare come chiave primaria un ID univoco invece del codice fiscale, per riuscire a gestire anche i casi di omocodia.

\clearpage
\subsection{Eliminazione degli identificatori esterni}
Dallo schema principale sono state eliminate le seguenti relazioni:
\begin{itemize}
    \item \textbf{Specificazione}, importando Tipo in Polizza
    \item \textbf{Stipulazione}, importando ID\_Assicurato in Polizza
    \item \textbf{Erogazione}, importando Denominazione in Polizza
    \item \textbf{Coinvolgimento}, importando ID\_Assicurato in Sinistro
    \item \textbf{Generazione}, importando Denominazione in Sinistro
    \item \textbf{Categorizzazione}, importando Appellativo in Sinistro
    \item \textbf{Avvenimento}, importando ID\_Luogo in Sinistro
    \item \textbf{Residenza}, importando ID\_Luogo in Studio Peritale
    \item \textbf{Delegazione}, importando ID\_Studio in Sinistro
    \item \textbf{Supervisione}, importando ID\_Studio in Supervisore
    \item \textbf{Assunzione}, importando ID\_Studio in Perito
    \item \textbf{Assegnazione}, importando ID\_Supervisore in Incarico
    \item \textbf{Ricezione}, importando ID\_Perito in Incarico
    \item \textbf{Fascicolo}, importando ID\_Incarico in Documento
    \item \textbf{Annessione}, importando ID\_Incarico in Video\_Perizia
    \item \textbf{Allegato}, importando NumeroPerizia ID\_Incarico in Media
\end{itemize}
\clearpage
\section{Analisi delle ridondanze}
Si è deciso di inserire ID\_Assicurato sia in Documento sia in Video\_Perizia, questo perché sono molto frequenti le operazioni che associano i vari documenti e video-perizie, alla parte coinvolta. Di seguito è mostrata la differenza tra i due casi, in termini di accessi, in entrambe le operazioni.
\\
\subsection{11 - Visualizzare quale assicurato ha svolto una determinata video-perizia}

\textbf{Senza ridondanza}
\\
\\
\def\arraystretch{2}% 
\begin{tabularx}{\textwidth}{ >{\centering\arraybackslash}p{3cm} | >{\centering\arraybackslash}X | >{\centering\arraybackslash}X |  >{\centering\arraybackslash}X }
    \textbf{Concetto} & \textbf{Costrutto} & \textbf{Accessi} & \textbf{Tipo} \\
    \hline
    Video-perizia & E & 1 & L \\
    Annessione & A & 1 & L \\
    Incarico & E & 1 & L \\
    Sinistro & E & 1 & L \\
    Coinvolgimento & A & 1 & L \\
    Assicurato & E & 1 & L \\
\end{tabularx}
\begin{center}
\textbf{Totale : 6L * 15.000 al giorno = 90.000 al giorno}
\end{center}

\textbf{Con ridondanza}
\\
\\
\def\arraystretch{2}% 
\begin{tabularx}{\textwidth}{ >{\centering\arraybackslash}p{3cm} | >{\centering\arraybackslash}X | >{\centering\arraybackslash}X |  >{\centering\arraybackslash}X }
    \textbf{Concetto} & \textbf{Costrutto} & \textbf{Accessi} & \textbf{Tipo} \\
    \hline
    Video-perizia & E & 1 & L \\
\end{tabularx}
\begin{center}
\textbf{Totale : 1L * 15.000 al giorno = 15.000 al giorno}
\end{center}

\clearpage

\subsection{14 - Visualizzare il proprietario (assicurato) di un documento}

\textbf{Senza ridondanza}
\\
\\
\def\arraystretch{2}% 
\begin{tabularx}{\textwidth}{ >{\centering\arraybackslash}p{3cm} | >{\centering\arraybackslash}X | >{\centering\arraybackslash}X |  >{\centering\arraybackslash}X }
    \textbf{Concetto} & \textbf{Costrutto} & \textbf{Accessi} & \textbf{Tipo} \\
    \hline
    Documento & E & 1 & L \\
    Fascicolo & A & 1 & L \\
    Incarico & E & 1 & L \\
    Sinistro & E & 1 & L \\
    Coinvolgimento & A & 1 & L \\
    Assicurato & E & 1 & L \\
\end{tabularx}
\begin{center}
\textbf{Totale : 6L * 15.000 al giorno = 90.000 al giorno}
\end{center}

\textbf{Con ridondanza}
\\
\\
\def\arraystretch{2}% 
\begin{tabularx}{\textwidth}{ >{\centering\arraybackslash}p{3cm} | >{\centering\arraybackslash}X | >{\centering\arraybackslash}X |  >{\centering\arraybackslash}X }
    \textbf{Concetto} & \textbf{Costrutto} & \textbf{Accessi} & \textbf{Tipo} \\
    \hline
   Documento & E & 1 & L \\
\end{tabularx}
\begin{center}
\textbf{Totale : 1L * 15.000 al giorno = 15.000 al giorno}
\end{center}

\clearpage

\section{Traduzione di entità e associazioni in relazioni}

\textbf{Assicurati}(\underline{ID\_Assicurato}, Nome, Cognome, DataNascita, CodiceFiscale, Telefono, Email)
\\
\\
\textbf{Polizze}(\underline{Numero}, \underline{Tipo}, \underline{Assicurato}, \underline{Assicurazione} Massimale, Costo, Scadenza) \\
FK: Tipo \texttt{REFERENCES} Tipi\_Polizze  \\
FK: Assicurato \texttt{REFERENCES} Assicurati \\
FK: Assicurazione \texttt{REFERENCES} Assicurazioni \\
\\
\textbf{Tipi\_Polizze}(\underline{Tipo}, Descrizione)
\\
\\
\textbf{Assicurazioni}(\underline{Denominazione}, Telefono, NumeroVerde, Email) \\
\\
\textbf{Categorie\_Sinistri}(\underline{Categoria}, Descrizione) \\
\\
\textbf{Sinistri}(\underline{ID\_Sinistro}, \underline{Assicurazione}, Assicurato, Categoria, Luogo, Studio\_Peritale, Descrizione) \\
FK: Assicurazione \texttt{REFERENCES} Assicurazioni \\
FK: Assicurato \texttt{REFERENCES} Assicurati \\
FK: Categoria \texttt{REFERENCES} Categorie\_Sinistri  \\
FK: Luogo \texttt{REFERENCES} Luoghi \\
FK: Studio\_Peritale \texttt{REFERENCES} Studio\_Peritali \\
\\
\textbf{Luoghi}(\underline{ID\_Luogo}, Via, Civico, Comune, Provincia, Citta, CAP) \\
\\
\textbf{Studi\_Peritali}(\underline{ID\_Studio}, Luogo, Nome, Telefono, Email) \\
FK: Luogo \texttt{REFERENCES} Luoghi \\
\\
\textbf{Supervisori}(\underline{ID\_Supervisore}, Nome, Cognome, DataNascita, CodiceFiscale, Telefono, Email) \\
FK: Studio \texttt{REFERENCES} Studi\_Peritali \\
\\
\textbf{Periti}(\underline{ID\_Perito}, Nome, Cognome, DataNascita, CodiceFiscale, Telefono, Email) \\
FK: Studio \texttt{REFERENCES} Studi\_Peritali \\
\clearpage
\noindent
\textbf{Incarichi}(\underline{ID\_Incarico}, Perito, Supervisore, ID\_Sinistro, Stato) \\
FK: Perito \texttt{REFERENCES} Periti \\
FK: Supervisore \texttt{REFERENCES} Supervisori \\
\\
\textbf{Documenti}(\underline{ID\_Documento}, Incarico, Tipo) \\
FK: Incarico \texttt{REFERENCES} Incarichi \\
\\
\textbf{Video\_Perizie}(\underline{NumeroPerizia}, \underline{Incarico}, Durata) \\
FK: Incarico \texttt{REFERENCES} Incarichi \\
\\
\textbf{Media}(\underline{NumeroMedia}, \underline{Video\_Perizia}, \underline{Incarico}, Tipo, Nome, Estensione, Dimensione, Metadati) \\
FK: Incarico \texttt{REFERENCES} Incarichi \\
FK: Video\_Perizia \texttt{REFERENCES} Video\_Perizie \\

\section{Schema relazionale finale}
\section{Traduzione delle operazioni in query SQL}

\chapter{Progettazione dell'applicazione}


Descrizione con screen

\end{document}
